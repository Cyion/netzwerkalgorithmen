\section{Minimale aufspannende Bäume}
\begin{problem}[minimaler aufspannender Baum / kürzestes zsgd.es Netzwerk]~\\[5pt]
	\hspace*{10pt}\textbf{Gegeben: } Ein Graph $G = (V, E)$, Gewichtfunktion
	\begin{eqnarray*}
		c : E &\to& \reell_+\\
		e &\mapsto& c_e
	\end{eqnarray*}
	\hspace*{10pt}\textbf{Gesucht: } Eine Kantenmenge $F\subseteq E$ mit möglichst geringen Gesamtgewicht $\sum_{e \in F} c_e$, \hspace*{60pt}so dass in $T = (V, F)$ alle Knoten verbunden sind ODER entscheide, dass \hspace*{56pt} $G$ nicht zshgd. ist.
\end{problem}\vspace*{10pt}
\begin{enumerate}
	\item Welche Eigenschaften haben Lösungen?\\
	$\hookrightarrow$Struktur
	\item Wie findet man optimale Lösungen?
	\item Können wir algorithmische Grundideen verwenden, die auch in anderen Situationen funktionieren?
	\item Effizienz
\end{enumerate}
\begin{beobachtung*}
	Ein optimale Lösung für Problem 2.1 kann kein Kreis enthalten. (Beweisskizze: Entfernt man aus einem Kreis eine Kante bleibt der Rest immer noch verbunden).
\end{beobachtung*}
\begin{satz}
	Eine optimale Lösung für Problem 2.1 ist zusammenhängend und kreisfrei, also ein Baum.
\end{satz}
\begin{proof}
	Klar.
\end{proof}
Wir betrachten Eigenschaften von Bäumen. Wie viele Kanten hat ein Baum mit $n$ Knoten? Mit jeder eingefügten Kante nimmt die Zahl der Zhk. um 1 ab (mit jeder gelöschten um 1 zu).
\begin{beobachtung*}
	Ein Baum mit $n$ Knoten hat $n-1$ Kanten. ABER: Nicht jeder Graph mit $n-1$ Kanten und $n$ Knoten ist ein Baum.
\end{beobachtung*}
\begin{definition}
	Für einen Graphen $G$ und $X,Y\subseteq V(G)$ ist
	\begin{eqnarray*}
		E(X,Y) &=& \set{\set{x,y}\in E(G)\vert x \in X\setminus Y,~y \in Y\setminus X}\\
		E^+(X,Y) &=& \set{(x,y)\in E(G)\vert x \in X\setminus Y,~y \in Y\setminus X}
	\end{eqnarray*}
	Für einen ungerichteten Graphen $G$ und $X \subseteq V(G)$
	\begin{equation*}
	\delta(X) = E(X, V(G)\setminus X)
	\end{equation*}
	Für einen gerichteten Graphen $G$ und $X\subseteq V(G)$
	\begin{eqnarray*}
		\delta^+(X) &=& E^+(X, V(G)\setminus X)\\
		\delta^-(X) &=& \delta^+(V(G)\setminus X)\\
		\delta(X) &=& \delta^+(X) \cup \delta^-(X)
	\end{eqnarray*}
\end{definition}
\begin{lemma}~
	\begin{enumerate}[a)]
		\item Ein ungerichteter Graph $G$ ist zusammenhängend $\Leftrightarrow \delta(X) \neq \emptyset~ \forall ~\emptyset \neq X \subset V(G)$.
		\item Sei $G$ ein gerichteter Graph und $r \in V(G)$. Dann gibt es einen Pfad von $r$ nach $v$ für jeden $v \in V(G) \Leftrightarrow \delta^+(X) \neq \emptyset~\forall~X\subset V(G)$ mit $r\in X$.
	\end{enumerate}
\end{lemma}
\begin{proof}~
	\begin{enumerate}[a)]
		\item Gibt es eine Menge $X\subset V(G)$ mit $r\in X,~v\in V(G)\setminus X$ und $\delta(X) = \emptyset$ dann kann es keinen $r$-$v$-Pfad geben, damit ist $G$ nicht zusammenhängend. Wenn $G$ nicht zusammenhängend ist, dann gibt es für irgendwelche Knoten $r$ und $v$ keinen $r$-$v$-Pfad. Sei $R$ die Menge der Knoten, die von $r$ aus erreichbar sind. Es gilt $r\in R,~v\notin R$ und $\delta(R)=\emptyset$.
		\item H.A.
	\end{enumerate}
\end{proof}
\begin{satz}[Eigenschaften von Bäumen]
	Sei $T=(V,F)$ ein Graph mit $n$ Knoten. Dann sind äquivalent
	\begin{enumerate}[a)]
		\item $T$ ist ein Baum (zshgd., kreisfrei).
		\item $T$ hat $n-1$ Kanten und ist kreisfrei.
		\item $T$ hat $n-1$ Kanten und ist zshgd..
		\item $T$ ist ein minimaler zusammenhängender Graph (D.h. keine Kante kann entfernt werden ohne dass der Zusammenhang verloren geht).
		\item $T$ ist ein minimaler Graph, für den es für jede Knotenmenge $\emptyset \neq X\subseteq V$ eine nach außen verbindende Kante gibt $\delta(X) \neq \emptyset$.
		\item $T$ ist ein maximaler kreisfreier Graph (D.h. keine Kante kann hinzugefügt werden ohne dass ein Kreis entsteht).
		\item $T$ enthält einen eindeutigen Pfad zwischen je zwei Knoten.
	\end{enumerate}
\end{satz}
\begin{proof}
	Wir zeigen a) $\Rightarrow$ g) $\Rightarrow$ e) $\Rightarrow$ d) $\Rightarrow$ f) $\Rightarrow$ b) $\Rightarrow$ c) $\Rightarrow$ a) (Ringschluss).\\
	a) $\Rightarrow$ g): Weil $T$ zusammenhängend ist, muss es je einen Pfad zwischen zwei Knoten geben. Gebe es zwischen zwei Knoten mehr als einen Pfad, so enthielte die Vereinigung der Pfade einen Kreis $\lightning$ Kreisfreiheit.\\
	g) $\Rightarrow$ e) $\Rightarrow$ d): Folgt aus Lemma 2.1. a).\\
	d) $\Rightarrow$ f): Da alle Knoten bereits verbunden sind, liefert jede eingefügte Kante einen Kreis.
	f) $\Rightarrow$ b) $\Rightarrow$ c): Ist klar, wenn man Lemma 2.2 zeigt.
	c) $\Rightarrow$ a): Sei $T$ zusammenhängend mit $n-1$ Kanten. Solange Kreise in $T$ existieren löschen wir eine der Kanten des Kreises und entfernen somit den Kreis. Angenommen wir haben $k$ Kanten entfernt. Der entstehende Graph $T'$ ist immer noch zusammenhängend und kreisfrei. Wir haben $p=1$ Komponenten und $m=(n-1)-k$ Kanten $\overset{Lemma~2.2}{\Rightarrow} n =m+p=n-1-k+1 \Rightarrow k=0$. 
\end{proof}
\begin{lemma}
	$W$ sei ein Wald mit $n$ Knoten, $m$ Kanten und $p$ ZhK.. Dann gilt $n=m+p$.
\end{lemma}
\begin{proof}
	H.A.
\end{proof}
\subsection{Berechnung optimaler Bäume}
\subsection*{Kruskals Algorithmus}
\begin{algorithm}
	\Input{Zshgd.er, ungerichteter Graph $G$\\Kantengewichte $c:E(G) \to \reell_+$}
	\Output{Aufspannender Baum $T$ minimalen Gewichts}\vspace*{5pt}
	Sortiere Kanten nach Gewicht\newline $c(e_1)\le c(e_2)\le ... \le c(e_m)$\\
	Setze $T:=(V(G), \emptyset)$\\
	\For{i=1 \KwTo m}
	{
	\uIf{$T+e_i$ enthält keinen Kreis}{$T:=T+e_i$}
	}
	\caption{Kruskals Algorithmus}
	\label{fig:Algorithmus}
\end{algorithm}
\subsubsection*{Laufzeit}
\begin{enumerate}
	\item Sortieren der Kanten $O(m~log(m)) = \underbrace{O(m~log(n)}_{\text{Da } m \in O(n^2)}$.
	\item Initialisierung des Baumes $O(1)$.
	\item Test auf Kreisfreiheit $O(m~log(n))$.
\end{enumerate}
\begin{problem}~\\[5pt]
	\hspace*{10pt}\textbf{Gegeben: } $G'$ kreisfrei ($\le n-1$ Kanten), Kante $e$.\\[5pt]
	\hspace*{10pt}\textbf{Frage: } Ist $G' + e$ kreisfrei?
\end{problem}
DFS oder BFS $\to O(n)$.\\
$\hookrightarrow$ Teste das $m$-fach $\to O(n\cdot m)$\\
\textbf{Aber} mit anderer Implementierung geht es besser.
\begin{satz}
	Kruskals Algorithmus kann so implementiert werden, dass sich eine Laufzeit von $O(m~log(n))$ ergibt.
\end{satz}
\begin{proof}
	Wir löschen parallele Kanten (nur die Billigste bleibt, die Anderen sind redundant). Für 3. Datenstruktur, die die ZhKs von $T$ verwaltet. Test in 3. ob $T+e_i = \set{v,w}$ ergibt Kreis $\Leftrightarrow$ $v,w$ liegen in derselben ZhK (Kante verbindet zwei Knoten in derselben ZhK). Für jede Komponente merken wir uns einen gerichteten Baum mit einer eindeutigen Wurzel, wobei jeder Knoten einen eindeutigen Vorgänger hat. Wenn wir $e_i=\set{v,w}$ in 3. prüfen, finden wir die Wurzel $r_v$ und $r_w$ zu den ZhKs von $B$, die $v$ bzw. $w$ enthalten. Zeit? Proportional zu Länge $r_v$-$v$-Pfad in $B + r_w$-$w$-Pfad in $B$. $\hookrightarrow O(log(n))$ ($=$ Höhe der Bäume) (noch zu zeigen). Teste $r_v = r_w$.
	\begin{itemize}
		\item falls ja: überprüfe nächste Kante.
		\item falls nein: wir fügen $e_i$ zu $T$ hinzu und wir müssen eine Kante zu $B$ hinzufügen.
	\end{itemize}
	Sei $h(r)$ die maximale Länge eines Pfades von $r$ in $B$.
	\begin{itemize}
		\item falls $h(r_v)\ge h(r_w)$: füge Kante $(r_v, r_w)$ zu B hinzu.
		\item sonst füge $(r_w, r_v)$ zu $B$ hinzu.
	\end{itemize}
	Änderung von $h(r_v)$?
	\begin{itemize}
		\item $h(r_v) = h(r_w) \Rightarrow$ erhöht $h(r_v)$ um eins.
		\item sonst: neue Wurzel hat gleichen $h$-Wert wie vorher.
	\end{itemize}
	Behauptung: Ein gerichteter Teilbaum von $B$ mit Wurzel $r$ enthält mindestens $2^{h(r)}$ Knoten. Beweis durch Induktion:\\
	I:A.: $B=(V(G), \emptyset),~h(r)=0$. Beh. gilt. Zu zeigen: Eigenschaft bleibt erhalten, wenn wir eine Kante $(x,y)$ zu $B$ hinzufügen.
	\begin{itemize}
		\item wenn $h(r)$ sich nicht ändert.
		\item sonst: vorher hatten wir $h(r_x) = h(r_y)$. Beide ZhKs von $B$ hatten mindestens $2^{h(r_x)}$ Knoten $\Rightarrow$ neue ZhK, verwurzelt in $r_x$, hat mind. $2\cdot2^{h(r_x)} = 2^{h(r_x)+1}$ Knoten. 
	\end{itemize}
	$\Rightarrow h(r) \le log(n) \Rightarrow \text{log. Höhe} \Rightarrow$ Laufzeit $O(m~log(n))$.
\end{proof}
\subsubsection*{Korrektheit}
Wir schauen uns zwei Optimalitätsbedingungen an.	
\begin{satz}
	Sei $(G,c)$ eine Instanz des Minimum Spanning Tree Problems (MST), sei $T$ ein Spannbaum in $G$. Dann gilt
	\begin{enumerate}[a)]
		\item $T$ ist optimal.
		\item Für jede Kante $e=\set{x,y} \in E(G)\setminus E(T)$ hat keine Kante auf dem $x$-$y$-Pfad in $T$ höhere Kosten als $e$.
		\item Für jede Kante $e\in E(T)$ ist $e$ eine billigste Kante in $\delta(V(C))$, wobei $C$ eine ZhK von $T-e$ ist.
	\end{enumerate}
\end{satz}
\begin{proof}~\\
	a)$\Rightarrow$ b): Wir nehmen an b) ist verletzt. Sei $e=\set{x,y} \in E(G)\setminus E(T)$ und $f$ eine Kante aus dem $x$-$y$-Pfad in $T$ mit $c(f) > c(e)$. Dann ist $(T-f)+e$ ein Spannbaum mit niedrigeren Kosten als $T$ $\lightning$ $T$ optimal.\\
	b)$\Rightarrow$ c): Wir nehmen an c) ist verletzt. Sei $e\in E(T)$, $C$ eine ZhK von $T-e$ und $f=\set{x,y} \in \delta(V(C))$ mit $c(f) < c(e)$. Der $x$-$y$-Pfad in $T$ muss eine Kante in $\delta(V(C))$ enthalten, aber die einzige Kante, die dies erfüllt ist $e$.\\
	c) $\Rightarrow$ a): Sei $T$ ein Baum, der c) erfüllt. Sei $T^*$ ein optimaler MST mit $E(T^*)\cap E(T)$ maximal. Wir zeigen $T=T^*$. Angenommen, es gibt eine Kante $e=\set{x,y} \in E(T)\setminus E(T^*)$. Sei $C$ eine ZhK von $T-e$. $T^*+e$ enthält einen Kreis $D$. Da $e\in E(D)\cap \delta(V(C))$ muss mindestens noch eine weitere Kante $f (\neq e)$ aus $D$ zu $\delta(V(C))$ gehören. $(T^*+e)-f$ ist ein Spannbaum. $T^*$ optimal $\Rightarrow c(e)\ge c(f)$. c) gilt für $T \Rightarrow c(f)\ge c(e)\Rightarrow c(f) = c(e)$. $\Rightarrow (T^*+e)-f$ ist ein anderer optimaler MST $\lightning$ (Widerspruch $E(T^*)\cap E(T)$ maximal (hat eine Kante mehr gemeinsam mit $T$)).
\end{proof}
\begin{satz}
	Kruskals Algorithmus ist korrekt.
\end{satz}
\begin{proof}
	Der Algorithmus konstruiert einen Spannbaum (einen maximalen kreisfreien Teilgraphen $\overset{Satz~2.2~f)}{\Rightarrow}$ Spannbaum). Zudem wird Bedingung b) aus Satz 2.4 erfüllt $\Rightarrow T$ ist optimal.
\end{proof}
Kruskal nutzt Optimalitätskriterium 2.4 b) - können wir auch c) nutzen?\\
\hspace*{10pt}$\hookrightarrow$ Ja! $\to$ Prims Algorithmus
\subsection*{Prims Algorithmus}
\begin{algorithm}
	\Input{Zshgd.er, ungerichteter Graph $G$\\Kantengewichte $c:E(G) \to \reell$}
	\Output{Spannbaum $T$ minimalen Gewichts}\vspace*{5pt}
	Wähle $v\in V(G)$\\
	Setze $T:=(\set{v}, \emptyset)$\\
	\While{$V(T) \neq V(G)$}
	{
		Wähle eine Kante $e\in \delta_G(V(T))$ mit minimalen Gewicht\\
		Setze $T=T+e$
	}
	\caption{Prims Algorithmus}
	\label{fig:Algorithmus}
\end{algorithm}
\begin{satz}
	Prims Algorithmus ist korrekt. Die Laufzeit ist $O(n^2)$.
\end{satz}
\begin{proof}
	Korrektheit: folgt, da Bedingung c) aus 2.4 erfüllt ist. Laufzeit: Nicht in 3. je alle Kanten druchsehen $\to O(n^2)$. Sondern: für jeden knoten $v\in V(T)$ merken wir uns die billigste Kante (\dq Kandidatenkante\dq) $e\in E(V(T), \set{v})$.
	\begin{itemize}
		\item Initialisierung der Kandidatenkanten $O(m)$
		\item Auswahl der billigsten Kante unter den Kandidaten $O(n)$
		\item Update der Kandidaten: überprüfe alle Kanten die zum zu $V(T)$ neu hinzugefügten Knoten inzident sind $\to O(n)$
	\end{itemize}
	WHILE-Schleife in 3. hat $n-1$ Iterationen $\Rightarrow O(n^2)$.
\end{proof}
\begin{rem}~
	\begin{enumerate}
		\item Laufzeit $O(n^2)$ ist bestmöglich für dichte Graphen ($m\in \Omega(n^2)$) da wir jede Kante ansehen müssen
		\item mit Fibonacci heaps kann $O(m+n~log(n))$ erreicht werden [für $m\in O(n)$ noch nicht besser]
		\item bester det. Algorithmus von Chazelle (2000) $O(m\cdot \underbrace{\alpha(m,n)}_{\mathclap{\text{inverse Ackermannfunktion}}})$
		\item Es existiert noch kein det. Algorithmus mit Laufzeit $O(m)$
		\item Bekannt:
		\begin{itemize}
			\item Randomisierte Algorithmen mit erwarteter Laufzeit $O(m)$
			\item Planare Graphen in $O(m)$
			\item $n$ Punkte in der Ebene $O(n~log(n))$
		\end{itemize}
	\end{enumerate}
\end{rem}
	