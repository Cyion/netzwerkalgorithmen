\section{Matching}
\begin{definition}[Matching]~
	\begin{itemize}
		\item[i)] Ein \textbf{Matching} in einem Grap $G=(V,E)$ ist eine Menge paarweise disjunkter Kanten.
		\item[ii)] Ein Matching heißt \textbf{perfekt}, wenn jeder Knoten in $V(G)$ zu einer Matchingkante gehört.
		\item[iii)] Ein \textbf{Vertex Cover} ist eine Kantenüberdeckende Knotenmenge.
		\item[iv)] \textbf{Maximales Matching} in $G:$ $\mathcal{V}(G)$.
		\item[v)] \textbf{Minimales Vertex Cover} in $G:$ $\mathcal{T}(G)$.
	\end{itemize}
\end{definition}
\begin{problem}[Kardinalitätsmatching/Max Matching]~\\[5pt]
	\hspace*{10pt}\textbf{Gegeben: }Ungerichteter Graph $G = (V,E)$.\\[5pt]
	\hspace*{10pt}\textbf{Gesucht: }Ein Matching $M$ größtmöglicher Kardinalität.
\end{problem}
\begin{problem}[Minimales Vertex Cover]~\\[5pt]
	\hspace*{10pt}\textbf{Gegeben: }Ungerichteter Graph $G = (V,E)$.\\[5pt]
	\hspace*{10pt}\textbf{Gesucht: }Ein Vertex Cover $C$ kleinstmöglicher Kardinalität.
\end{problem}
\begin{satz}
	Sei $G$ ein ungerichteter Graph. Es gilt max Matching $\le$ min Vertex Cover.
\end{satz}
\begin{definition}[Bipartite Graphen]
	Ein Graph heißt \textbf{bipartit}, wenn sich die Knotenmenge so in zwei Mengen $A,B$ mit $V(G) = A \dot\cup B$ zerlegen lässt, dass jede Kante genau einen Knoten in $A$ und einen Knoten in $B$ enthält.
\end{definition}
\textit{Damit: Bipartite Graphen enthalten keine Kreise ungerader Länge. Für bipartite Graphen gibt es eine Beziehung zu Flussproblemen.}
\begin{satz}
	Ein bipartites Matching maximaler Kardinalität in $G$ entspricht einem maximalen Fluss in $(G',u,s,t)$ und umgekehrt.
\end{satz}
\begin{proof}~
	\begin{itemize}
		\item Jedes Matching in $G$ lässt sich direkt auf einen Fluss in $G'$ abbilden, indem die Matchingkanten einen Fluss von jeweisl 1 erhalten und entsprechende Flüsse von $s$ und $t$ gewählt werden.
		\item Umgekehrt lässt sich jeder ganzzahliger Fluss (der auf jeder Kante den Wert 0 oder 1 hat) auf ein Matching in $G$ abbilden.
	\end{itemize}
	Da es immer ein ganzzahliges Optimum für das Flussproblem gibt, sind also insbesondere die Optimalwerte gleich.
\end{proof}
\begin{satz}
	In bipartiten Graphen gilt $\mathcal{V}(G) = \mathcal{T}(G)$.
\end{satz}
\begin{proof}
	
\end{proof}
\begin{satz}[Satz von Hall]
	Sei $G$ ein bipartiter Graph mit $V(G)=A\dot\cup B$. Dann hat $G$ ein $A$ überdeckendes Matching $\Leftrightarrow$ $\card{\underbrace{T(X)}_{\mathclap{\text{Menge der Nachbarn von } X}}} \ge \card{X}~\forall X\subseteq A$.
\end{satz}
\begin{proof}
	
\end{proof}
\begin{korollar}[Heiratssatz von Frobenius]
	Sei $G$ ein bipartiter Graph mit $V(G) = A\dot\cup B$. Dann hat $G$ ein perfektes Matching $\Leftrightarrow \card{A}=\card{B}$ und $\card{T(X)} \ge \card{X}~\forall X \subseteq A$.
\end{korollar}
\textit{Aus dem Beweis von Satz 5.7 folgt}
\begin{korollar}
	Das Kardinalitätsmatching-Problem kann in bipartiten Graphen in $O(n\cdot m)$ gelöst werden.
\end{korollar}
\begin{proof}
	Konstruktion von oben. Betrachte Ford-Fulkerson für das äquivalente Flussproblem
	\begin{itemize}
		\item Eine Augmentierung benötigt $O(m)$.
		\item Um einen maximalen $s$-$t$-Fluss (und damit ein maximales Matching) zu finden brauchen wir höchstens $n$ Augmentierungen $\Rightarrow$ $(O(m\cdot n))$.
	\end{itemize}
\end{proof}
\textit{Wie sehen Augmentierungen aus?\\
\hspace*{10pt}$\hookrightarrow$ Verbesserung von Matchings\\
Augmentierende Pfade in $(G',u,s,t)$ entsprechen alternierenden Pfaden in $G$.}
\begin{definition}
	Sei $G$ ein Graph (bipartit oder nicht), und sei $M$ ein beliebiges Matching in $G$. Ein Pfad $P$ ist ein \textbf{$M$-alternierender Pfad}, wenn $E(P)\setminus M$ ein Matching ist. Ein $M$-alternierender Pfad ist \textbf{$M$-augmentierend}, wenn seine Endpunkte nicht von $M$ überdeckt werden (er zwei nicht gematchte Knoten verbindet).
\end{definition}
\textit{Augmentierende Pfade müssen ungerade Länge haben.}
