\section{Kürzeste Wege}
\begin{problem}~\\[5pt]
	\hspace*{10pt}\textbf{Gegeben: }Gerichteter Graph $G$, Gewichte $c:A(G)\to \reell$, zwei ausgezeichnete \hspace*{60pt} Knoten $s,~t$.\\[5pt]
	\hspace*{10pt}\textbf{Gesucht: }$s$-$t$-Pfad minimalen Gesamtgewichts.
\end{problem}
\textit{Verschiedene Schwierigkeitsstufen
\begin{itemize}
	\item alle Kanten Einheitslänge $c(a)=1~ \forall a\in A(G)$.\\
	$\hookrightarrow$ (BFS) Suche nach Pfad mit minimaler Anzahl an Kanten.
	\item nicht negative Kantenlängen.\\
	$\hookrightarrow$ Dijkstra.
	\item ohne Kreise negativen Gesamtgewichts.
	\item allgemeine Kantengewichte.
\end{itemize}}
\begin{definition}[Konservativ]
	Sei $G$ ein Graph mit Gewichten $c:E(G)\to \reell$. Diese Kantengewichtsfunktion heißt \textbf{konservativ} wenn sie keine Kreise negativen Gesamtgewichts enthält.
\end{definition}
\subsection{Kürzeste Wege von einer Quelle}
\textit{Algorithmen beruhen auf einer Beobachtung \dq Bellmans Prinzip\dq.}
\begin{lemma}
	Sei $G$ ein Digraph mit konservativen Kantengewichten $c:A(G)\to \reell$, seinen $s$ und $w$ zwei Knoten. Ist $e=(v,w)$ die letzte Kante eines kürzesten Pfades $P$ von $s$ nach $w$, dann ist $P_{[s,v]}$ ($P$ ohne die Kante $e$) ein kürzester Pfad von $s$ nach $v$.
\end{lemma}
\begin{proof}
	Angenommen $Q$ ist ein kürzerer $s$-$v$-Pfad als $P_{[s,v]} \Rightarrow c(Q) + c(e) < c(P)$.
	\begin{itemize}
		\item Falls $Q$ $w$ nicht enthält $\Rightarrow Q + e$ ist kürzerer $s$-$w$-Pfad als $P$.
		\item Sonst gilt für $Q_{[s,w]}$:\\ $c(Q_{[s,w]}) = c(Q) + c(e) - c(Q_{[w,v]} + e) < c(P) - \underbrace{c(Q_{[w,v]} + e)}_{\mathclap{\text{ist ein Kreis und $c$ ist konservativ}}}) \le c(P)$\\ $\lightning$ $P$ ist kürzester $s$-$w$-Pfad.
	\end{itemize}
\end{proof}
\textit{Folgerungen:
\begin{itemize}
	\item Algorithmen bauen Wege schrittweise auf.
	\item Grund, dass meiste Algorithmen den kürzesten Weg von $s$ zu allen Knoten berechnen: Kürzesten $s$-$t$-Pfad berechnet $\Rightarrow$ auch kürzesten $s$-$v$-Pfad $\forall v\in P$ + wir wissen vorher nicht welche Knoten zu $P$ gehören.
\end{itemize}
Wir betrachten zunächst nicht-negative Kantengewichte. Sind diese auch ganzzahlig könnten wir auch BFS nutzen, indem wir Knoten durch Einheitskanten ersetzen.\\
Nachteil: Bläst Eingabe um exponentiellen Faktor auf. Statt
\begin{eqnarray*}
	&&\Theta(\underbrace{n~log(m)}_{\mathclap{\text{inzidente Kanten}}} + \underbrace{m~log(n)}_{Kanten} + \sum_{\mathclap{e \in E(G)}} log(c(e))) \text{ bekommen wir}\\
	&&\Theta(n'~log(m') + m'~log(n')) \text{ mit } m' = \sum_{\mathclap{e \in E(G)}} c(e),~n' = n + \sum_{\mathclap{e \in E(G)}} c(e) - 1
\end{eqnarray*}
Besser: Dijkstra.}
\begin{definition}
	Sei $G$ ein Digraph mit $s, v \in V(G)$. Wir definieren
	\begin{itemize}
		\item $L(v) :=$ Länge eines kürzesten $s$-$v$-Pfades
		\item $p(v) :=$ Vorgänger von $v$ in einem kürzesten $s$-$v$-Pfad
	\end{itemize}
	($v$ nicht erreichbar: $L(v) = \infty,~p(v) = \text{NIL}$).
\end{definition}
\begin{algorithm}
	\Input{Digraph $G$, Gewichte $c:A(G)\to \reell_+$, Knoten $s\in V(G)$}
	\Output{Kürzeste Wege von $s$ zu allen $v\in V(G)$ und ihre Länge, genauer $L(v),~p(v)$}\vspace*{5pt}
	Setze $L(s) = 0$\\
	\hspace*{25pt}$L(v) = \infty~\forall v \in V(G)\setminus \set{s}$\\
	\hspace*{25pt}$R = \emptyset$\\
	Finde einen Knoten $v\in V(G)\setminus R$ mit $L(v) = \underset{\mathclap{w\in V(G)\setminus R}}{min}~L(w)$\\
	Setze $R = R \cup \set{v}$\\
	\For{$\forall w \in V(G)\setminus R$ mit $(v,w) \in A(G)$}{
		\If{$L(w)>L(v) + c((v,w))$}{
			Setze $L(w) = L(v) + c((v,w))$\\
			$p(w) = v$
		}
	}
	\If{$R\neq V(G)$}{\textbf{goto} 4}
	\caption{Dijkstras Algorithmus}
	\label{fig:Algorithmus}
\end{algorithm}
\begin{satz}
	Dijkstras Algorithmus ist korrekt. Die Laufzeit ist $O(n^2)$.
\end{satz}
\begin{proof}
	Wir zeigen, dass die folgenden Aussagen nach jeder Ausführung von 4. gelten
	\begin{enumerate}[label=\alph*)]
		\item $\forall v\in R$ und $w\in V(G)\setminus R: L(v) \le L(w)$.
		\item $\forall v\in R: L(v)$ ist die Länge eines kürzesten $s$-$v$-Pfades in $G$. Wenn $L(v) < \infty$, dann existiert ein $s$-$v$-Pfad der Länge $L(v)$, dessen letzte Kante $(p(v), v)$ ist und dessen Knoten alle zu $R$ gehören.
		\item $\forall w\in V(G)\setminus R: L(w)$ ist die Länge eines kürzesten $s$-$w$-Pfades in $G[R\cup \set{w}]$. Wenn $L(w) \neq \infty$, dann $p(w)\in R$ und $L(w) = L(p(w)) + c((p(w),w))$.
	\end{enumerate}
	Die Aussagen gelten nach 1-3. Wir müssen zeigen, dass 5 und 6 die Gültigkeit von a), b) und c) aufrecht erhalten. Sei $v$ ein in 4 ausgewählter Knoten.\\
	zu a): Für jedes $x \in R,~y\in V(G)\setminus R$ gilt nach a) und der Wahl von $v$ in 4: $L(x) \le L(v) \le L(y) \Rightarrow$ a) gilt weiter nach 5 und 6.\\
	zu b): Zu zeigen: b) gilt nach 5 (dazu reicht: betrachte Knoten $v$). c) galt vor 5 $\to$ reicht zu zeigen, dass kein $s$-$v$-Pfad in $G$, der irgendein Knoten in $V(G)\setminus R$ enthält, kürzer ist als $L(v)$. Wir nehmen an es gebe einen $s$-$v$-Pfad $P$ in $G$, der einen knoten $w \in V(G)\setminus R$ enthält, der kürzer als $L(v)$ ist. Sei $w$ der erste Knoten außerhalb von $R$, wenn wir $P$ von $s$ nach $v$ durchlaufen. c) galt vor 5 $\Rightarrow L(w) \le c(P_{[s,w]})$. Kantengewichte nicht-negativ $\Rightarrow c(P_{[s,w]}) \le c(P) < L(v) \Rightarrow L(w) < L(v)~ \lightning$ Wahl von $v$ in 4.\\
	zu c): Wir zeigen, dass 5 und 6 c) aufrecht erhalten $\to$ wenn für $w$ (wir setzen $p(w) = v$ und $L(w) = L(v) + c((v,w))$) in 6 $\to$ existiert ein $s$-$v$-Pfad in $G[R \cup \set{w}]$ der Länge $L(v) + c((v,w))$ mit letzter Kante $(v,w)$ + wir wissen c) galt für $v$. Wir nehmen an, dass nach 5 und 6 ein $s$-$v$-Pfad $P$ in $G[R \cup \set{w}]$ existiert, der kürzer als $L(w)$ für ein $w\in V(G)\setminus R$ ist. $P$ muss $v$ enthalten - nur $w$ wurde hinzugefügt und sonst wäre c) schon vor 5 und 6 verletzt gewesen (wir wissen $L(w)$ wächst nicht). Sei $x$ der Nachbar von $w$ in $P$: $x \in R \overset{a)}{\Rightarrow} L(x) \le L(v) \overset{6}{\Rightarrow} L(w) \le L(x) + c((x,w)) \Rightarrow L(w) \le L(x) + c((x,w)) \le L(v) + c((x,w)) \le c(P) ~\lightning$ b) $\to$ $L(v)$ ist Länge eines kürzesten $s$-$v$-Pfades und $P$ enthält
	\begin{itemize}
		\item einen $s$-$v$-Pfad
		\item Kante $(x,w)$
	\end{itemize}
	Wir haben gezeigt:
	\begin{itemize}
		\item a), b) und c) gelten je in 4
		\item b) gilt insbesondere, wenn der Algorithmus terminiert\\
		$\hookrightarrow$ Ausgabe korrekt.
	\end{itemize}
	Laufzeit:
	\begin{itemize}
		\item $n$ Iterationen
		\item je $O(n)$
		\item[$\to$] $O(n^2)$
	\end{itemize}
\end{proof}
\begin{rem}~
	\begin{itemize}
		\item $O(n^2)$ wieder bestmöglich für dichte Graphen.
		\item für \dq dünne\dq~Graphen: Friedman und Tarjan (1987) haben mittels Fibonacci heaps die Laufzeit auf $O(n~log(n) + m)$ verbessert. Beste Laufzeit für nicht-negative Gewichte.
	\end{itemize}
\end{rem}
\textit{Jetzt: konservative Gewichte.}
\begin{algorithm}
	\Input{Digraph $G$, konservative Gewichte $c:A(G)\to \reell$, Knoten $s\in V(G)$}
	\Output{Kürzeste Wege von $s$ zu allen $v\in V(G)$ und ihre Länge}\vspace*{5pt}
	Setze $L(s) = 0$\\
	\hspace*{25pt}$L(v) = \infty~\forall v \in V(G)\setminus \set{s}$\\
	\For{$i=1$ \textbf{to} $n-1$}{
		\For{jede Kante $(v,w)\in A(G)$}{
			\If{$L(w)>L(v) + c((v,w))$}{
				Setze $L(w) = L(v) + c((v,w))$\\
				$p(w) = v$
			}
		}
	}
	\caption{Moore-Bellman-Ford Algorithmus}
	\label{fig:Algorithmus}
\end{algorithm}
\begin{satz}
	Der Moore-Bellman-Ford Algorithmus ist korrekt und hat eine Laufzeit von $O(m\cdot n)$.
\end{satz}
\begin{proof}
	Sei zu beliebigen Zeitpunkt $R := \set{v\in V(G) ~\vert~ L(v) < \infty}$, $F := \set{(x,y)\in A(G)~\vert~x=p(y)}$. Wir behaupten:
	\begin{enumerate}[label=\alph*)]
		\item $L(y) \ge L(x) + c((x,y)) ~\forall (x,y)\in F$.
		\item Wenn $F$ einen Kreis $C$ enthält hat dieser negatives Gesamtgewicht.
		\item Wenn $C$ konservativ ist, ist $(R,F)$ eine in $s$ verwurzelte Arboreszenz.
	\end{enumerate}
	zu a): $L(y) = L(x) + c((x,y))$ wenn $p(y) = x$ gesetzt wurde und $L(x)$ wird nie erhöht.\\
	zu b): Wir nehmen an, es wird ein Kreis $C$ in $F$ erzeugt indem wir $p(y) = x$ setzen $\to$ vor Einfügen: $L(y) > L(x) + c((x,y)) \Rightarrow L(y) - L(x) > c((x,y))$ und wegen a) $L(w) \ge L(v) + c((v,w))~\forall (v,w) \in A(G)\setminus \set{(x,y)} \Rightarrow L(w) - L(v) \ge c((v,w))$. Aufsummieren $\to$ Gesamtgewicht von $C$ negativ $c((x,y)) + c((y, v_1)) + ... + c((v_l,x)) < L(y) - L(x) + L(v_1) - L(y) + ... + L(x) - L(v_l) = 0$.\\
	zu c): $C$ konservativ $\overset{b)}{\Rightarrow}$ $F$ azyklisch. $x\in R\setminus \set{s} \Rightarrow p(x) \in R \Rightarrow (R,F)$ Arboreszenz die in $s$ verwurzelt.\\
	$L(x)$ ist die länge des $s$-$x$-Pfades in $(R,F)$ für jeden $x\in R$ (zu bel. Zeitpunkt im Algorithmus). Wir behaupten nach $k$ Iterationen, $L(x)$ ist höchstens so lang wie ein kürzester $s$-$x$-Pfad mit $k$ Kanten. Beweis durch Induktion:\\
	Sei $P$ ein kürzester $s$-$x$-Pfad mit höchstens $k$ Kanten und sei $(w,x)$ die letzte Kante von $P$ $\Rightarrow$ $P_{[s,w]}$ ist kürzester $s$-$w$-Pfad mit höchstens $k-1$ Kanten $\overset{I.V.}{\Rightarrow} L(w) \le c(P_{[s,w]})$ nach $k-1$ Iterationen. In $k$-ter Iteration wird auch Kante $(w,x)$ betrachtet, danach $L(x) \le L(w) + c((w,x)) \le c(P)$. Da kein Weg mehr als $n-1$ Kanten hat folgt aus der Behauptung die Korrektheit vom Algorithmus.\\
	Laufzeit: offensichtlich.
\end{proof}
\begin{rem}
	Dieser Algorithmus ist der Schnellste für konservative Gewichte.
\end{rem}