\section{Netzwerkflüsse}
\textit{Flüsse in Graphen. Beispiele:
\begin{itemize}
	\item Wasser in Leitungssystemen
	\item Verkehr in Straßensystemen
	\item Passagiere in Transportsystemen
\end{itemize}
Eigenschaften:
\begin{itemize}
	\item pro Kante 2 Kenngrößen
	\begin{itemize}
		\item mögliche Flusskapazität
		\item tatsächlicher Fluss
	\end{itemize}
	\item \dq Flusserhaltung\dq : Bilanzierung der Flüsse an jedem Knoten (\dq fliest nicht mehr raus, als rein\dq)
\end{itemize}}
\begin{definition}
	Gegeben: Digraph $G$ mit Kapazität $u: A(G) \to \reell_+$.
	\begin{enumerate}
		\item Ein \textbf{Fluss} ist eine Funktion $f: A(G) \to \reell_+$ mit $f(e) \le u(e)~\forall e\in A(G)$.
		\item An einem Knoten $v$ gilt \textbf{Flusserhaltung}, wenn \[\sum_{e\in \delta^-(v)} f(e) = \sum_{e\in \delta^+(v)} f(e).\]
		\item Eine \textbf{Zirkulation} ist ein Fluss für den an jedem Knoten Flusserhaltung gilt.
		\item Für ein Netzwerk $(G, u, s, t)$ ist ein Fluss ein $s$-$t$-Fluss, wenn Flusserhaltung an allen Knoten außer $s$ und $t$ gilt \[Wert(f) :=\sum_{e\in \delta^+(s)} f(e) - \sum_{e\in \delta^-(s)} f(e).\]
	\end{enumerate}
\end{definition}
\begin{problem}[Maximaler Fluss]~\\[5pt]
\hspace*{10pt}\textbf{Gegeben: }Netzwerk $(G, u, s, t)$.\\[5pt]
\hspace*{10pt}\textbf{Gesucht: }Ein $s$-$t$-Fluss mit maximalem Wert.
\end{problem}
\textit{Formulierung als LP (Lineares Programm):
\begin{eqnarray*}
	max~ F \text{ sodass} &&\sum_{e\in \delta^+(s)} f_e - \sum_{e\in \delta^-(s)} f_e = F,\\
	&&\sum_{e\in \delta^-(v)} f_e - \sum_{e\in \delta^+(v)} f_e = 0~\forall v\in V(G)\setminus \set{s,t},\\
	&&0\le f_e \le u_e~\forall e\in A(G).
\end{eqnarray*}
Also: Können wir effizient lösen (LP $\in$ P).\vspace*{5pt}\\
Ziel hier: \dq kombinatorische\dq Algorithmen.}
\begin{beobachtung}
	Das Max Flow Problem hat immer eine optimale Lösung.
\end{beobachtung}
\begin{proof}~
	\begin{enumerate}
		\item LP ist beschränkt.
		\item Fluss mit $f\equiv 0$ ist immer zulässig.
	\end{enumerate}
\end{proof}
\begin{definition}
	In einem Digraph $G$ heißt eine Kantenmenge $\delta^+(X)$ mit $s\in X$ und $t\notin X$ ein \textbf{$s$-$t$-Schnitt}.
	Wert des Schnittes := Summe der Kantenkapazitäten der ausgehenden Kanten.
\end{definition}
\begin{lemma}[Schranken durch Schnitte]
	Für bel. $A\subseteq V(G)$ mit $s\in A$, $t\notin A$ und jedem $s$-$t$-Fluss $f$ gilt
	\begin{enumerate}[label=\alph*)]
		\item $Wert(f) = \sum_{e\in \delta^-(A)} f(e) - \sum_{e\in \delta^+(A)} f(e)$
		\item $Wert(f) \le \sum_{e\in \delta^+(A)} f(e)$
	\end{enumerate}
\end{lemma}
\begin{proof}
	zu a): Flusserhaltung gilt für $v\in A\setminus \set{s}$, daraus folgt
	\begin{eqnarray*}
		Wert(f) &=& \sum_{e\in \delta^+(s)} f(e) - \sum_{e\in \delta^-(s)} f(e)\\
		&=& \sum_{v\in A}(\sum_{e\in \delta^+(v)} f(e) - \sum_{e\in \delta^-(v)} f(e))\\
		&=& \sum_{e\in \delta^+(A)} f(e) - \sum_{e\in \delta^-(A)} f(e).
	\end{eqnarray*}
	zu b): Folgt mit a), wenn wir $0 \le f(e) \le u(e)~\forall e\in A(G)$ nutzen.
\end{proof}
\textit{Was sagt Lemma 4.5:\\[5pt]
Der Wert eines Max Flow kann die Kapazität eines minimalen $s$-$t$-Schnitts nicht überschreiben - tatsächlich gilt Gleichheit, zeigen wir später.\\[5pt]
Grundkonzept für Algorithmen: Flüsse erhöhen oder verringern. Interpretation: Fluss verringern $\widehat{=}$ Fluss umleiten.}
\begin{definition}~
	\begin{enumerate}
		\item Für ein Digraph $G$ ist $\overset{\leftrightarrow}{G} := (V(G), A(G) \dot\cup \set{\overset{\leftarrow}{e} \vert e \in A(G)})$ mit $\overset{\leftarrow}{e} = (w,v)$ für $e=(v,w)$ \dq Rückwärtskante\dq~ von $e$ der doppelt gewichtete Graph für $G$.
		\item Für einen Fluss $f$ in einem Digraph $G$ mit Kapazität $z:A(G)\to \reell_+$ definieren wir die \textbf{Residualkapazitäten} $u_f$ \[ u_f(e) = \begin{cases} u(e) - f(e) &\text{für Vorwärtskante,}\\ f(e) &\text{für Rückwärtskante.} \end{cases} \]
		\item Der \textbf{Residualgraph} $G_t$ ist der Graph $(V(G), \set{e\in A(\overset{\leftrightarrow}{G})~\vert~ u_f(e) > 0})$.
		\item Interpretation des Residualgraphen:
		\begin{itemize}
			\item $u_f$ auf Vorwärtskante: um wie viel wir $f$ noch erhöhen können.
			\item $u_f$ auf Rückwärtskante: um wie viel wir $f$ noch verringern können.
		\end{itemize}
	\item Gegeben ein Fluss $f$ und ein Pfad (oder Kreis) $P$ in $G_f$. Um $f$ entlang $P$ um $\gamma$ zu augmentieren muss man
	\begin{itemize}
		\item den Fluss für jede Vorwärtskante um $\gamma$ erhöhen ($e\in A(P)$: wenn $e \in A(G)$, erhöhe $f(e)$ um $\gamma$)
		\item den Fluss für jede Rückwärtskante um $\gamma$ reduzieren ($e \in A(P)$: wenn $e = \overset{\leftarrow}{e_0}$ für $e_0 \in A(G)$, reduziere $f(e_0)$ um $\gamma$)
	\end{itemize}
	\item Ein \textbf{$f$-augmentierender} Pfad in einem Netzwerk $(G, u, s, t)$ mit einem Fluss $f$ ist ein $s$-$t$-Pfad im Residualgraph $G_f$.
	\end{enumerate}
\end{definition}
\begin{algorithm}
	\Input{Netzwerk $(G, u, s, t)$}
	\Output{Ein $s$-$t$-Fluss $f$ mit maximalem Wert}\vspace*{5pt}
	Setze $f(e)=0~\forall e\in A(G)$\\
	Finde einen $f$-augmentierenden Pfad $P$\\
	\hspace*{15pt}Falls keiner existiert: \textbf{stop} \\
	Berechne $\gamma := \underset{e\in P}{min}~ u_f(e)$\\
	\hspace*{15pt}Augmentiere $f$ entlang $P$ um $\gamma$ und \textbf{goto} 2
	\caption{Ford-Fulkerson Algorithmus}
	\label{fig:Algorithmus}
\end{algorithm}
\begin{satz}
	Ein $s$-$t$-Fluss ist optimal $\Leftrightarrow$ Es gibt keinen $f$-augmentierenden Pfad.
\end{satz}
\textit{Also: stoppt der Algorithmus, dann ist $f$ tatsächlich ein optimaler Fluss.}
\begin{proof}~\\
	$\Rightarrow$: Wenn es einen $f$-augmentierenden Pfad gibt $\to$ 4 berechnet Fluss mit größerem Wert $\Rightarrow$ $f$ war nicht maximal.\\
	$\Leftarrow$: Wenn es keinen augmentierenden Pfad gibt $\to$ $t$ ist in $G_f$ von $s$ nicht erreichbar. $R:= \set{\text{Knoten in } G_f \text{ , die von } s \text{ erreichbar sind}}$, Definition von $G_f$:\\ \begin{equation*}
		f(e) = \begin{cases}
		u(e) & \forall e \in \delta_G^+(R),\\
		0 & \forall e \in \delta_G^-(R).
		\end{cases}
	\end{equation*}
	$\overset{\text{4.5 a)}}{\Rightarrow} Wert(f) = \sum_{e \in \delta^+(R)} u(e) \overset{\text{4.5 b)}}{\Rightarrow} f$ ist optimal.
\end{proof}
\subsection{Schwierigkeit mit Ford-Fulkerson}
\begin{satz}[MaxFlow-MinCut]
	In einem Netzwerk $(G, u, s, t)$ ist der maximale Wert eines $s$-$t$-Flusses gleich der minimalen Kapazität eines $s$-$t$-Schnittes.
\end{satz}
\begin{proof}
	Satz 4.7 $\Rightarrow$ für jeden maximalen $s$-$t$-Fluss $f$ gibt es einen $s$-$t$-Schnitt dessen Kapazität gleich dem Flusswert ist $\Rightarrow~max~Wert(f) \ge~min~\text{Wert eines Schnittes} \overset{4.5~b)}{\Rightarrow}$ MaxFlow = MinCut. $Wert(f) \le \sum_{e\in \delta^+(A)}u(e)$
\end{proof}
\begin{beobachtung*}
	Wollen wir den Wert eines maximalen Flusses erhöhen, müssen wir am minimalen Cut die Kapazitäten verändern.
\end{beobachtung*}
\begin{korollar}
	Wenn die Kapazitäten in einem Netzwerk ganzzahlig sind, dann existiert ein optimaler ganzzahliger Fluss.
\end{korollar}
\begin{satz}
	Sei $(G, u, s, t)$ ein Netzwerk und $f$ ein $s$-$t$-Fluss in $G$. Dann gibt es
	\begin{itemize}
		\item eine Familie $P$ von $s$-$t$-Pfaden
		\item eine Familie $C$ von Kreisen
	\end{itemize}
	so dass \[f(e) = \sum_{p\in P \cup C: e \in E(p)}w(p), ~\forall~e\in E(G)\] und $\card{P} + \card{C}\le \card{E(G)}$. $\sum_{p\in P} w(p) = Wert(f)$. \\[5pt]
	Außerdem: Wenn $f$ ganzzahlig ist kann $w$ ganzzahlig gewählt werden.
\end{satz}
\begin{proof}
	Wir konstruieren $P, C$ und $w$ durch Induktion über die Anzahl der Kanten mit positivem Fluss. Sei $e=(v_0, w_0)$ eine Kante mit $f(e)>0$. Falls $w_0\neq t$ gibt es eine Kante $(v_0, w_1)$ mit positivem Fluss (Flusserhaltung). Wir machen immer weiter (mit $i=1$ startend) und finden ein $w_{i+1}$:\\
	Wenn $w_i\in \set{t, v_0, w_0, ..., w_{i-1}}$ stoppen wir (Muss passieren, da endliche Knotenzahl). Sonst finden wir Kante $(w_i, w_{i+1})$ mit positivem Fluss und setzen $i = i+1$.\\
	Das Gleiche machen wir in der anderen Richtung. Wir suchen sukzessive Vorgängerkanten: Falls $v_0 \neq s$, gibt es eine Kante $(v_1, v_0)$ mit positivem Fluss ... . Ende:
	\begin{enumerate}[label=\arabic*)]
		\item $w_i\in \set{v_0, w_0, ..., w_{i-1}}$ oder $v_j \in \set{w_0, v_0, ..., v_{j-1}}$ dann haben wir einen Kreis.
		\item $w_i=t$ und $v_j=s$, dann haben wir einen $s$-$t$-Pfad.
	\end{enumerate}
	Wir haben eine Kreis oder einen $s$-$t$-Pfad in $G$ gefunden. UND: wir haben nur Kanten mit positivem Fluss verwendet. Sei $p$ dieser Kreis oder Pfad. Sei $w(p) = \underset{e\in E(p)}{min}f(e)$.\\
	Setze \[f'(e):=\begin{cases}
		f(e) - w(p) &\text{für } e\in E(p),\\
		f(e) &\text{für } e\notin E(p).
	\end{cases}\]
	Das heißt wir entfernen einen Fluss in Höhe von $w(p)$ auf $p$. Dann können wir für $f'$ die Induktionsannahme verwenden (weniger Kanten mit positivem Fluss).
\end{proof}
\newpage
\begin{algorithm}
	\Input{Netzwerk $(G, u, s, t)$}
	\Output{Ein $s$-$t$-Fluss $f$ mit maximalem Wert}\vspace*{5pt}
	Setze $f(e)=0~\forall e\in A(G)$\\
	Finde einen kürzesten $f$-augmentierenden Pfad $P$\\
	\hspace*{15pt}Falls keiner existiert: \textbf{stop} \\
	Berechne $\gamma := \underset{e\in P}{min}~ u_f(e)$\\
	\hspace*{15pt}Augmentiere $f$ entlang $P$ um $\gamma$ und \textbf{goto} 2
	\caption{Edmonds-Karp Algorithmus (1972)}
	\label{fig:Algorithmus}
\end{algorithm}
\textit{Also: Implementiere 2. von Ford-Fulkerson als BFS in $G_f$}
\begin{lemma}
	Sei $f_1, f_2, ...$ eine Folge von Flüssen, wobei $f_{i+1}$ aus $f_i$ durch Augmentierung entlang eines kürzesten f-augmentierenden Pfades $P_i$ entsteht. Dann gilt:
	\begin{enumerate}[label=\alph*)]
		\item $\card{E(P_k)} \le \card{E(P_{k+1})} ~\forall~k$
		\item $\card{E(P_k)} + 2 \le \card{E(P_l)} ~\forall~k<l$, so dass $P_k \cup P_l$ ein Paar entgegengesetzer Kanten enthält.
	\end{enumerate}
\end{lemma}
\begin{proof}
	
\end{proof}
\begin{satz}
	Unabhängig von den Kantenkapazitäten stoppt Algorithmus 10 (Edmonds-Karp) nach höchstens $\frac{m\cdot n}{2}$ Augmentierungen.
\end{satz}
\begin{proof}
	
\end{proof}
\begin{korollar}
	Der Edmonds-Karp Algorithmus löst MaxFlow in $O(m^2\cdot n)$.
\end{korollar}
\begin{proof}
	
\end{proof}